%%%%%%%%%%%%%%%%%%%%%%%%%%%%%%%%%%%%%%%%%
% Short Sectioned Assignment
% LaTeX Template
% Version 1.0 (5/5/12)
%
% This template has been downloaded from:
% http://www.LaTeXTemplates.com
%
% Original author:
% Frits Wenneker (http://www.howtotex.com)
%
% License:
% CC BY-NC-SA 3.0 (http://creativecommons.org/licenses/by-nc-sa/3.0/)
%
%%%%%%%%%%%%%%%%%%%%%%%%%%%%%%%%%%%%%%%%%

%----------------------------------------------------------------------------------------
%	PACKAGES AND OTHER DOCUMENT CONFIGURATIONS
%----------------------------------------------------------------------------------------

\documentclass[paper=a4, fontsize=11pt]{scrartcl} % A4 paper and 11pt font size

\usepackage{mathpazo}
\usepackage{placeins}
\usepackage[T1]{fontenc} % Use 8-bit encoding that has 256 glyphs
\usepackage{fourier} % Use the Adobe Utopia font for the document - comment this line to return to the LaTeX default
\usepackage[english]{babel} % English language/hyphenation
\usepackage{amsmath,amsfonts,amsthm} % Math packages
\usepackage{listings}
\usepackage{amssymb}
\usepackage{graphicx}

\usepackage{sectsty} % Allows customizing section commands
\allsectionsfont{\raggedright \normalfont} % Make all sections centered, the default font and small caps

\usepackage{fancyhdr} % Custom headers and footers
\pagestyle{fancyplain} % Makes all pages in the document conform to the custom headers and footers
\fancyhead{} % No page header - if you want one, create it in the same way as the footers below
\fancyfoot[L]{} % Empty left footer
\fancyfoot[C]{} % Empty center footer
\fancyfoot[R]{\thepage} % Page numbering for right footer
\renewcommand{\headrulewidth}{0pt} % Remove header underlines
\renewcommand{\footrulewidth}{0pt} % Remove footer underlines
\setlength{\headheight}{9pt} % Customize the height of the header
\setlength{\textheight}{640pt} % Customize the height of the header

\setlength\parindent{0pt} % Removes all indentation from paragraphs - comment this line for an assignment with lots of text

%----------------------------------------------------------------------------------------
%	TITLE SECTION
%----------------------------------------------------------------------------------------

\newcommand{\horrule}[1]{\rule{\linewidth}{#1}} % Create horizontal rule command with 1 argument of height

\title{	
\normalfont \normalsize 
\textsc{Bonn-Rhein-Sieg University of Applied Sciences \\Department of Computer Science\\Learning and Adaptivity} \\ [10pt] % Your university, school and/or department name(s)
\horrule{0.5pt} \\[0.4cm] % Thin top horizontal rule
\LARGE  L\&A Project Proposal\\ % The assignment title
\horrule{2pt} \\[0.5cm] % Thick bottom horizontal rule
}
\date{2016-04-24}
\author{Padmaja Kulkarni, Benjamin Thompson} % Author(s)

\begin{document}

\maketitle % Print the title

%----------------------------------------------------------------------------------------
%	PROBLEM 1
%----------------------------------------------------------------------------------------
\section{Objective}

\subsection{What is it you would like to use Machine Learning to do? What area of ML does it fall into? (e.g. prediction, classification)}

\subsection{Why are you interested in this? Why should others be interested in this? Where would you take this if you had more than 5 weeks?}

\section{Data}

\subsection{What data are you going to use? Is the data labelled?}

\subsection{How much data is available?}

\subsection{How large will your training set be?}

\subsection{How large will your test set be?}

\section{Method}

\subsection{Which Machine Learning algorithm or Method do you plan to use? Which Programming Language Library do you plan to use? List 4 other projects which have used or applied this method.}

\section{Features}

\subsection{Which features are available in the dataset you plan to use?}

\subsection{What do you need to do to the data to get addition features if required? How long will it take? Do you have experience using the tools required to extract additional features?}

\section{Milestones}

\subsection{What will you be able to present after 1 week?}

\subsection{After two weeks, three weeks?}

\subsection{For the final deliverable, how will you evaluate your systems performance?}

\subsection{You will be expected to demonstrate a running system by the end of the course, what do you plan to present?}

%\FloatBarrier

%\begin{itemize}
%\item 
%\end{itemize}

%\begin{enumerate}
%\item
%\end{enumerate}

%\begin{figure}[h]
%\centering
%\includegraphics[width=0.4\textwidth]{!!! File Name !!!}
%\caption{!!! Caption !!!}
%\end{figure}

%$$\tiny \mathrm{math\ text:\ } X_Y = (10 \times 5) + \left[\frac{1}{2}\right] + 2^3 + a \cdot b \normalsize$$
\end{document}